% Ejemplo DE PEC en castellano.
\documentclass[IB]{PlantillaPACnova_Est}
\usepackage{hyperref} %url amb \ref. 
\usepackage{float}
\usepackage{amsmath}
\usepackage{hyperref}
\usepackage{url}
\def\vi{\vec{\i} }
\def\vj{\vec{\j} }
\def\vk{\vec{k} }
\def\vk{\vec{k} }
\def\vB{\vec{B} }
\def\vH{\vec{H} }
\def\vM{\vec{M} }
\def\vE{\vec{E} }
%\def<d{/>{\rm d}}
\def\vr{\vec{r}}
\def\Ohm{\Omega}
\def\qpi{\frac{1}{4\pi\varepsilon_0}}
\DeclareMathSymbol{,}{\mathord}{letters}{"3B}
%\usepackage{siunitx}
%\DeclareSIUnit\grau{\degreeCelsius}
%\sisetup{output-decimal-marker = {,}}
%\sisetup{per-mode = symbol}%
%\sisetup{exponent-product = \cdot}
%\usepackage{upgreek}
%\usepackage{cancel}
\setcounter{secnumdepth}{0} 
\usepackage[nottoc,numbib]{tocbibind}
\usepackage{tocloft}
\renewcommand{\cftsecleader}{\cftdotfill{\cftdotsep}}

\begin{document}
\textinicial
%%%%%%%%%%%%%%%%%%%%%%%%% Código %%%%%%%%%%%%%%%%%%%%%%%%%%%%%%%% --------------------> 1
{M2.851} 				
%%%%%%%%%%%%%%%%%%%%%%%%% Assignatura %%%%%%%%%%%%%%%%%%%%%%%%%%--------------------> 2
{Tipología y ciclo de vida de los datos}
%%%%%%%%%%%%%%%%%%%%%%%%% Número de PEC o Práctica %%%%%%%%%%%%%--------------------> 3
{PRA1}
%%%%%%%%%%%%%%%%%%%%%%%%% Curso i número de semestre %%%%%%%%%%%%--------------------> 4
{2021-22-Sem.2}
%%%%%%%%%%%%%%%%%%%%%%%%% Nombre Programa %%%%%%%%%%%%%%%%%%%%%%%%%--------------------> 5
{Master de ciencia de datos}
%%%%%%%%%%%%%%%%%%%%%%%%% Vuestro nombre %%%%%%%%%%%%%%%%%%%%%%%%%--------------------> 6
{Maria Dolores Moyano Guerrero y Víctor Cáncer Castillo}




\begin{center}
\textbf{{\LARGE PRA 1 - Tipología y ciclo de vida de los datos}}\\[1cm]

\textbf{{\Large Maria Dolores Moyano Guerrero y Víctor Cáncer Castillo}}
\end{center}

\tableofcontents

\section{Contexto}

Estos datos se han recogido para practicar el \textit{web scraping} en la asignatura de \textit{Tipología y ciclo de vida de los datos} del Máster de ciencia de datos de la UOC.\\

Cómo (futuros) científicos de datos hemos tenido la curiosidad por saber cómo está el mercado laboral actualmente en varias ciudades europeas y americanas. Además hemos querido averiguar en qué lugares el trabajo de científico de datos está más reconocido por las empresas y por lo tanto mejor remunerados. \\
Para ello hemos obtenido los suelos que se ofrecen por diferentes empresas utilizando la web \textit{Glassdoor} \cite{glassdoor}, donde los trabajadores pueden informar de su sueldo de manera anónima. Por otro lado hemos extraído datos de la web \textit{datosmacro.expansion.com} \cite{datosmacro} dónde hay múltiples datos económicos, entre ellos el salario medio, lo cual nos puede mostrar si el trabajo del científico de datos está mejor/peor remunerado que el resto de trabajos en ese pais o ciudad. 

\section{Título}

Obtendremos dos datasets de los cuales extraeremos nuestras conclusiones: 
\begin{itemize}
\item \textit{sueldos$\_$data$\_$scientist.csv}
\item \textit{salario$\_$medio.csv}
\end{itemize}

\section{Descripción del dataset}

--

\section{Representación gráfica}


--


\section{Contenido}



\section{Agradecimientos}



\section{Inspiración}


\section{Licencia}


\section{Código}

Para hacer tanto scraping como pre-procesado de datos hemos utilizado Python. El código está disponible en GitHub: \href{https://github.com/mdmoyano/web_scraping}.

\section{Dataset}

-- Publicar en Zenodo --

\section{Vídeo}

-- Link al video de cada uno --



%\bibliographystyle{unsrt}
%\bibliographystyle{alphadin}
\newpage
\bibliographystyle{plain}
\bibliography{biblio} 

\end{document}

